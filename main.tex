\documentclass[12pt,a4paper]{article}
\usepackage{fontspec}
\usepackage{amsmath,amssymb,amsthm}
\usepackage{geometry}
\usepackage{fancyhdr}
\usepackage{graphicx}
\usepackage{enumerate}
\usepackage{titlesec}
\usepackage{hyperref}
\usepackage{tcolorbox}
\usepackage{multicol}

\geometry{left=2cm,right=2cm,top=2.5cm,bottom=2.5cm}

\pagestyle{fancy}
\fancyhf{}
\fancyhead[L]{ĐỀ CƯƠNG ÔN TẬP GIỮA KÌ}
\fancyhead[R]{GIẢI TÍCH 1}
\fancyfoot[C]{\thepage}

\titleformat{\section}{\large\bfseries}{\thesection}{1em}{}
\titleformat{\subsection}{\normalsize\bfseries}{\thesubsection}{1em}{}

\newtheorem{definition}{Định nghĩa}[section]
\newtheorem{theorem}{Định lý}[section]
\newtheorem{example}{Ví dụ}[section]

\title{\textbf{ĐỀ CƯƠNG ÔN TẬP GIỮA KÌ\\GIẢI TÍCH 1}}
\author{Kuma Shiomi}
\begin{document}

\maketitle

\newpage
\textbf{Lời mở đầu}
  
  -Các nhà toán học cổ đại (chủ yếu làm việc với hình học) đã luôn đau đầu vì hai bài toán: tìm tiếp tuyến của một đường cong bất kỳ, và tính diện tích dưới một đường cong. Archimedes đã có một số kết quả nổi bật với phương pháp vét cạn. Nhưng phải cho tới thế kỉ XVII, với đại số của Viéte, hình học giải tích của Descartes và Fermat cùng với mối quan tâm dâng cao về chuyển động của các thiên thể mới thúc đẩy mạnh mẽ việc khai thác mảnh đất hoang này với đỉnh cao là các công trình của Newton và Leibniz.
-Một cách tự nhiên để tiếp cận giải tích là thông qua hình học giải tích, tức là hình học với các toạ độ, phương trình thay vì các lập luận logic thuần tuý như trong hình học Euclid cổ điển. Cụ thể hơn, các đối tượng hình học như điểm, đường thẳng, đường cong,... sẽ được mô tả bởi các hàm số cùng phương trình qua đó ta có thể thực hiện các phép toán đại số.
-Sự ra đời và phát triển của giải tích xoay quanh hình học và vật lý với muôn vàn vấn đề thú vị mà có thể nói tóm gọn: Giải tích là toán học của sự thay đổi.
-Mục đích của tài liệu là để ghi lại một vài thu hoạch về bộ môn biên soạn bởi một sinh viên đại học trường VinUniversity, nên có thể sẽ có những sự sai sót, xin hãy thông cảm.
-Tài liệu gồm nội dung lý thuyết và bài tập phục vụ cho việc ôn tập học phần Giải tích I cho kì thi giữa kì tại Học viện Công nghệ Bưu chính Viễn thông hy vọng có thể giúp đỡ được phần nào trong việc ôn tập và chuẩn bị cho kì thi.
- Cấu trúc đề thi giữa kì như dự đoán của mình sẽ gồm:

\begin{itemize}
    \item 1 bài tìm giới hạn (lim)
    \item 1 bài đạo hàm bậc cao
    \item 1 bài tích phân
    \item 1 bài khai triển Taylor
    \item 1 bài dãy Fourier
\newpage
\tableofcontents

\section{CHƯƠNG I: GIỚI HẠN CỦA DÃY SỐ}

\subsection{Số thực}

\subsubsection{Các tính chất cơ bản của tập số thực}

\begin{definition}[Số vô tỉ]
Một số biểu diễn dưới dạng thập phân vô hạn không tuần hoàn, hay không thể biểu diễn dưới dạng tỉ số của hai số nguyên được gọi là số vô tỉ.
\end{definition}

\begin{definition}[Số thực]
Tất cả các số hữu tỉ và số vô tỉ tạo thành tập hợp số thực, ký hiệu là $\mathbb{R}$.
\end{definition}

\textbf{Tính chất 1:} Tập $\mathbb{R}$ là một trường giao hoán với hai phép cộng và nhân: $(\mathbb{R}, +, \cdot)$.

\textbf{Tính chất 2:} Tập $\mathbb{R}$ được xếp thứ tự toàn phần.

\textbf{Tính chất 3:} Tập $\mathbb{R}$ là một tập đầy.

\subsubsection{Giá trị tuyệt đối}

\begin{definition}[Giá trị tuyệt đối]
Giá trị tuyệt đối của số thực $x$, được ký hiệu $|x|$, là số thực không âm xác định như sau:
\[
|x| = \begin{cases}
x & \text{khi } x \geq 0\\
-x & \text{khi } x < 0
\end{cases}
\]
\end{definition}

\textbf{Tính chất của giá trị tuyệt đối:}
\begin{enumerate}
\item $\forall x \in \mathbb{R}: |x| = \max\{x, -x\}$
\item $|x| = 0 \Leftrightarrow x = 0$
\item $\forall x,y \in \mathbb{R}: |xy| = |x||y|$
\item $\forall x,y \in \mathbb{R}: |x + y| \leq |x| + |y|$ (bất đẳng thức tam giác)
\item $\forall x,y \in \mathbb{R}: ||x| - |y|| \leq |x - y|$
\end{enumerate}

\subsection{Số phức}

\subsubsection{Định nghĩa và các dạng số phức}

\begin{definition}[Số phức]
Cho $(x,y) \in \mathbb{R}^2$, một số biểu diễn dưới dạng $z = x + iy$, trong đó $i^2 = -1$ được gọi là một số phức. Tập các số phức được ký hiệu là $\mathbb{C}$.
\end{definition}

\begin{itemize}
\item $x = \text{Re}(z)$ được gọi là phần thực của $z$
\item $y = \text{Im}(z)$ được gọi là phần ảo của $z$
\item Môđun: $|z| = \sqrt{x^2 + y^2} = r \geq 0$
\item Argument: $\text{Arg}(z) = \theta$ với $\cos\theta = \frac{x}{|z|}$ và $\sin\theta = \frac{y}{|z|}$
\end{itemize}

\textbf{Các dạng biểu diễn số phức:}

\begin{enumerate}
\item \textbf{Dạng đại số (chính tắc):} $z = x + iy$
\item \textbf{Dạng lượng giác:} $z = r(\cos\theta + i\sin\theta)$
\item \textbf{Dạng mũ (Euler):} $z = re^{i\theta}$
\end{enumerate}

\subsubsection{Các phép toán trên tập số phức}

Cho $z_1 = x_1 + iy_1 = r_1(\cos\theta_1 + i\sin\theta_1)$ và $z_2 = x_2 + iy_2 = r_2(\cos\theta_2 + i\sin\theta_2)$

\textbf{1. Phép cộng:}
\[z_1 + z_2 = (x_1 + x_2) + i(y_1 + y_2)\]

\textbf{2. Phép trừ:}
\[z_1 - z_2 = (x_1 - x_2) + i(y_1 - y_2)\]

\textbf{3. Phép nhân:}
\begin{align*}
z_1 \cdot z_2 &= (x_1x_2 - y_1y_2) + i(x_1y_2 + x_2y_1)\\
&= r_1r_2[\cos(\theta_1 + \theta_2) + i\sin(\theta_1 + \theta_2)]
\end{align*}

\textbf{4. Phép chia:}
\[\frac{z_1}{z_2} = \frac{r_1}{r_2}[\cos(\theta_1 - \theta_2) + i\sin(\theta_1 - \theta_2)], \quad z_2 \neq 0\]

\textbf{5. Số phức liên hợp:}
\[\overline{z} = x - iy = r(\cos\theta - i\sin\theta)\]

\subsubsection{Công thức Moivre}

\begin{theorem}[Công thức Moivre]
Với mọi $n \in \mathbb{Z}$:
\[z^n = r^n(\cos n\theta + i\sin n\theta)\]
\end{theorem}

\textbf{Hệ quả - Công thức khai triển:}
\begin{align*}
\cos n\theta &= \cos^n\theta - C_n^2\cos^{n-2}\theta\sin^2\theta + \ldots\\
\sin n\theta &= C_n^1\cos^{n-1}\theta\sin\theta - C_n^3\cos^{n-3}\theta\sin^3\theta + \ldots
\end{align*}

\subsection{Dãy số thực}

\subsubsection{Định nghĩa dãy số}

\begin{definition}[Dãy số thực]
Một dãy số thực là một ánh xạ từ $\mathbb{N}$ vào $\mathbb{R}$:
\[u: \mathbb{N} \to \mathbb{R}, \quad n \mapsto u_n\]
Ký hiệu dãy số: $(u_n)$ hoặc $\{u_n\}_{n=1}^{\infty}$
\end{definition}

\subsubsection{Giới hạn của dãy số}

\begin{definition}[Dãy số hội tụ]
Dãy số $(u_n)$ được gọi là hội tụ đến giới hạn $a \in \mathbb{R}$ nếu:
\[\forall \varepsilon > 0, \exists N \in \mathbb{N}: \forall n \geq N \Rightarrow |u_n - a| < \varepsilon\]
Ký hiệu: $\lim_{n\to\infty} u_n = a$ hay $u_n \to a$ khi $n \to \infty$
\end{definition}

\begin{definition}[Dãy số phân kỳ]
Dãy số $(u_n)$ phân kỳ nếu nó không hội tụ đến một giới hạn hữu hạn.
\end{definition}

\textbf{Các trường hợp đặc biệt:}
\begin{itemize}
\item $\lim_{n\to\infty} u_n = +\infty$ nếu $\forall M > 0, \exists N: \forall n \geq N \Rightarrow u_n > M$
\item $\lim_{n\to\infty} u_n = -\infty$ nếu $\forall M > 0, \exists N: \forall n \geq N \Rightarrow u_n < -M$
\end{itemize}

\subsubsection{Tính chất của dãy hội tụ}

\begin{theorem}[Tính duy nhất của giới hạn]
Nếu dãy $(u_n)$ hội tụ thì giới hạn của nó là duy nhất.
\end{theorem}

\begin{theorem}[Tính bị chặn]
Nếu dãy $(u_n)$ hội tụ thì $(u_n)$ bị chặn.
\end{theorem}

\begin{theorem}[Các phép toán với giới hạn]
Cho $\lim_{n\to\infty} u_n = a$, $\lim_{n\to\infty} v_n = b$. Khi đó:
\begin{enumerate}
\item $\lim_{n\to\infty} (u_n \pm v_n) = a \pm b$
\item $\lim_{n\to\infty} (u_n \cdot v_n) = a \cdot b$
\item $\lim_{n\to\infty} \frac{u_n}{v_n} = \frac{a}{b}$ (nếu $b \neq 0$)
\item $\lim_{n\to\infty} |u_n| = |a|$
\end{enumerate}
\end{theorem}

\subsubsection{Dãy đơn điệu}

\begin{definition}[Dãy đơn điệu]
\begin{itemize}
\item Dãy $(u_n)$ tăng nếu $u_n \leq u_{n+1}, \forall n$
\item Dãy $(u_n)$ giảm nếu $u_n \geq u_{n+1}, \forall n$
\item Dãy $(u_n)$ đơn điệu nếu nó tăng hoặc giảm
\end{itemize}
\end{definition}

\begin{theorem}[Định lý Weierstrass]
Mọi dãy số đơn điệu và bị chặn đều hội tụ.
\end{theorem}

\subsubsection{Số e}

\begin{definition}[Số e]
Số e được định nghĩa là giới hạn:
\[e = \lim_{n\to\infty} \left(1 + \frac{1}{n}\right)^n \approx 2.718281828...\]
\end{definition}

\textbf{Tính chất:}
\begin{itemize}
\item $e$ là số vô tỉ
\item $e = 1 + \frac{1}{1!} + \frac{1}{2!} + \frac{1}{3!} + \ldots = \sum_{k=0}^{\infty} \frac{1}{k!}$
\item $2 < e < 3$
\end{itemize}

\subsubsection{Nguyên lý Cauchy}

\begin{theorem}[Nguyên lý Cauchy]
Dãy $(u_n)$ hội tụ khi và chỉ khi:
\[\forall \varepsilon > 0, \exists N \in \mathbb{N}: \forall m,n \geq N \Rightarrow |u_m - u_n| < \varepsilon\]
\end{theorem}

\subsection{Các giới hạn cơ bản cần nhớ}

\begin{tcolorbox}[title=Các giới hạn cơ bản]
\begin{align*}
&\lim_{n\to\infty} \frac{1}{n} = 0 \quad &\lim_{n\to\infty} \frac{1}{n^p} = 0 \quad (p > 0)\\
&\lim_{n\to\infty} \sqrt[n]{n} = 1 \quad &\lim_{n\to\infty} \sqrt[n]{a} = 1 \quad (a > 0)\\
&\lim_{n\to\infty} q^n = 0 \quad (|q| < 1) \quad &\lim_{n\to\infty} \left(1 + \frac{1}{n}\right)^n = e
\end{align*}
\end{tcolorbox}

\newpage

\section{CHƯƠNG II: HÀM SỐ MỘT BIẾN SỐ}

\subsection{Các khái niệm cơ bản về hàm số}

\subsubsection{Định nghĩa hàm số}

\begin{definition}[Hàm số]
Cho $X$ là tập con không rỗng của $\mathbb{R}$. Một ánh xạ $f$ từ $X$ vào $\mathbb{R}$ được gọi là hàm số của một biến số:
\[f: X \to \mathbb{R}, \quad x \mapsto f(x)\]
\begin{itemize}
\item $X$ gọi là tập xác định của $f$
\item $f(X)$ gọi là tập giá trị của $f$
\item $x$ gọi là đối số, $y = f(x)$ gọi là hàm số
\end{itemize}
\end{definition}

\subsubsection{Các loại hàm số đặc biệt}

\textbf{1. Hàm số chẵn và hàm số lẻ:}
\begin{itemize}
\item Hàm $f(x)$ chẵn: $f(-x) = f(x), \forall x \in X$
\item Hàm $f(x)$ lẻ: $f(-x) = -f(x), \forall x \in X$
\end{itemize}

\textbf{2. Hàm số tuần hoàn:}
\begin{itemize}
\item Hàm $f(x)$ tuần hoàn với chu kỳ $T > 0$ nếu:
\[f(x + T) = f(x), \forall x \in X\]
\end{itemize}

\textbf{3. Hàm số đơn điệu:}
\begin{itemize}
\item Hàm $f(x)$ tăng: $x_1 < x_2 \Rightarrow f(x_1) \leq f(x_2)$
\item Hàm $f(x)$ tăng ngặt: $x_1 < x_2 \Rightarrow f(x_1) < f(x_2)$
\item Hàm $f(x)$ giảm: $x_1 < x_2 \Rightarrow f(x_1) \geq f(x_2)$
\item Hàm $f(x)$ giảm ngặt: $x_1 < x_2 \Rightarrow f(x_1) > f(x_2)$
\end{itemize}

\textbf{4. Hàm số bị chặn:}
\begin{itemize}
\item Hàm $f(x)$ bị chặn trên: $\exists A: f(x) \leq A, \forall x \in X$
\item Hàm $f(x)$ bị chặn dưới: $\exists B: f(x) \geq B, \forall x \in X$
\item Hàm $f(x)$ bị chặn: $\exists M > 0: |f(x)| \leq M, \forall x \in X$
\end{itemize}

\subsection{Các hàm số sơ cấp cơ bản}

\subsubsection{Hàm lũy thừa}

\[y = x^\alpha, \quad \alpha \in \mathbb{R}\]

\subsubsection{Hàm mũ}

\[y = a^x, \quad a > 0, a \neq 1\]
Đặc biệt: $y = e^x$

\subsubsection{Hàm logarit}

\[y = \log_a x, \quad a > 0, a \neq 1\]
Đặc biệt: $y = \ln x$ (logarit tự nhiên)

\textbf{Tính chất:}
\begin{enumerate}
\item $\log_a 1 = 0$
\item $\log_a(xy) = \log_a x + \log_a y$
\item $\log_a\left(\frac{x}{y}\right) = \log_a x - \log_a y$
\item $\log_a x^\alpha = \alpha \log_a x$
\item $\log_b x = \log_b a \cdot \log_a x$
\end{enumerate}

\subsubsection{Hàm lượng giác}

\begin{itemize}
\item $y = \sin x$: Xác định trên $\mathbb{R}$, lẻ, tuần hoàn với chu kỳ $2\pi$, $-1 \leq \sin x \leq 1$
\item $y = \cos x$: Xác định trên $\mathbb{R}$, chẵn, tuần hoàn với chu kỳ $2\pi$, $-1 \leq \cos x \leq 1$
\item $y = \tan x$: Xác định trên $\mathbb{R} \setminus \{\frac{\pi}{2} + k\pi, k \in \mathbb{Z}\}$, lẻ, tuần hoàn với chu kỳ $\pi$
\item $y = \cot x$: Xác định trên $\mathbb{R} \setminus \{k\pi, k \in \mathbb{Z}\}$, lẻ, tuần hoàn với chu kỳ $\pi$
\end{itemize}

\subsubsection{Hàm lượng giác ngược}

\begin{itemize}
\item $y = \arcsin x$: $x \in [-1,1]$, $y \in [-\frac{\pi}{2}, \frac{\pi}{2}]$
\item $y = \arccos x$: $x \in [-1,1]$, $y \in [0, \pi]$
\item $y = \arctan x$: $x \in \mathbb{R}$, $y \in (-\frac{\pi}{2}, \frac{\pi}{2})$
\item $y = \text{arccot}\, x$: $x \in \mathbb{R}$, $y \in (0, \pi)$
\end{itemize}

\subsection{Giới hạn của hàm số}

\subsubsection{Định nghĩa giới hạn}

\begin{definition}[Giới hạn tại một điểm]
Cho $f: X \to \mathbb{R}$ và $a$ là điểm tụ của $X$. Ta nói $f$ có giới hạn là $L$ khi $x$ tiến tới $a$ nếu:
\[\forall \varepsilon > 0, \exists \delta > 0: 0 < |x - a| < \delta \Rightarrow |f(x) - L| < \varepsilon\]
Ký hiệu: $\lim_{x \to a} f(x) = L$
\end{definition}

\begin{definition}[Giới hạn một phía]
\begin{itemize}
\item Giới hạn phải: $\lim_{x \to a^+} f(x) = L$ nếu $\forall \varepsilon > 0, \exists \delta > 0: 0 < x - a < \delta \Rightarrow |f(x) - L| < \varepsilon$
\item Giới hạn trái: $\lim_{x \to a^-} f(x) = L$ nếu $\forall \varepsilon > 0, \exists \delta > 0: 0 < a - x < \delta \Rightarrow |f(x) - L| < \varepsilon$
\end{itemize}
\end{definition}

\begin{theorem}
$\lim_{x \to a} f(x) = L$ khi và chỉ khi $\lim_{x \to a^+} f(x) = \lim_{x \to a^-} f(x) = L$
\end{theorem}

\subsubsection{Các giới hạn đáng nhớ}

\begin{tcolorbox}[title=Các giới hạn cơ bản cần nhớ]
\begin{align*}
&\lim_{x \to 0} \frac{\sin x}{x} = 1 \quad &\lim_{x \to 0} \frac{\tan x}{x} = 1\\
&\lim_{x \to 0} \frac{1 - \cos x}{x^2} = \frac{1}{2} \quad &\lim_{x \to 0} \frac{e^x - 1}{x} = 1\\
&\lim_{x \to 0} \frac{\ln(1+x)}{x} = 1 \quad &\lim_{x \to 0} \frac{(1+x)^\alpha - 1}{x} = \alpha\\
&\lim_{x \to 0} (1+x)^{\frac{1}{x}} = e \quad &\lim_{x \to \infty} \left(1 + \frac{1}{x}\right)^x = e\\
&\lim_{x \to 0} \frac{a^x - 1}{x} = \ln a \quad (a > 0)
\end{align*}
\end{tcolorbox}

\subsection{Đại lượng vô cùng bé và vô cùng lớn}

\subsubsection{Đại lượng vô cùng bé (VCB)}

\begin{definition}[VCB]
Hàm $\alpha(x)$ được gọi là đại lượng vô cùng bé khi $x \to a$ nếu:
\[\lim_{x \to a} \alpha(x) = 0\]
\end{definition}

\textbf{So sánh các VCB:}
Cho $\alpha(x), \beta(x)$ là các VCB khi $x \to a$, $\beta(x) \neq 0$:
\begin{itemize}
\item Nếu $\lim_{x \to a} \frac{\alpha(x)}{\beta(x)} = 0$: $\alpha$ là VCB bậc cao hơn $\beta$, ký hiệu $\alpha = o(\beta)$
\item Nếu $\lim_{x \to a} \frac{\alpha(x)}{\beta(x)} = c \neq 0$: $\alpha$ và $\beta$ là VCB cùng bậc
\item Nếu $\lim_{x \to a} \frac{\alpha(x)}{\beta(x)} = 1$: $\alpha$ và $\beta$ là VCB tương đương, ký hiệu $\alpha \sim \beta$
\end{itemize}

\textbf{Các VCB tương đương thường dùng khi $x \to 0$:}
\begin{align*}
&\sin x \sim x, \quad \tan x \sim x, \quad \arcsin x \sim x, \quad \arctan x \sim x\\
&1 - \cos x \sim \frac{x^2}{2}, \quad e^x - 1 \sim x, \quad \ln(1+x) \sim x\\
&(1+x)^\alpha - 1 \sim \alpha x, \quad a^x - 1 \sim x\ln a
\end{align*}

\subsubsection{Đại lượng vô cùng lớn (VCL)}

\begin{definition}[VCL]
Hàm $f(x)$ được gọi là đại lượng vô cùng lớn khi $x \to a$ nếu:
\[\lim_{x \to a} f(x) = \infty\]
\end{definition}

\textbf{Quan hệ giữa VCB và VCL:}
Nếu $\alpha(x)$ là VCB khi $x \to a$ và $\alpha(x) \neq 0$ thì $\frac{1}{\alpha(x)}$ là VCL khi $x \to a$ và ngược lại.

\subsection{Sự liên tục của hàm số}

\subsubsection{Định nghĩa}

\begin{definition}[Hàm liên tục tại một điểm]
Hàm $f$ liên tục tại $a$ nếu:
\[\lim_{x \to a} f(x) = f(a)\]
hay tương đương:
\[\forall \varepsilon > 0, \exists \delta > 0: |x - a| < \delta \Rightarrow |f(x) - f(a)| < \varepsilon\]
\end{definition}

\begin{definition}[Hàm liên tục trên một khoảng]
Hàm $f$ liên tục trên khoảng $(a,b)$ nếu $f$ liên tục tại mọi điểm thuộc $(a,b)$.
\end{definition}

\subsubsection{Các tính chất}

\begin{theorem}[Phép toán với hàm liên tục]
Nếu $f$ và $g$ liên tục tại $a$ thì:
\begin{enumerate}
\item $f \pm g$ liên tục tại $a$
\item $f \cdot g$ liên tục tại $a$
\item $\frac{f}{g}$ liên tục tại $a$ (nếu $g(a) \neq 0$)
\end{enumerate}
\end{theorem}

\begin{theorem}[Tính chất của hàm liên tục trên đoạn]
Nếu $f$ liên tục trên $[a,b]$ thì:
\begin{enumerate}
\item $f$ bị chặn trên $[a,b]$
\item $f$ đạt giá trị lớn nhất và nhỏ nhất trên $[a,b]$
\item $f$ nhận mọi giá trị trung gian (Định lý giá trị trung gian)
\end{enumerate}
\end{theorem}

\begin{theorem}[Định lý giá trị trung gian - Bolzano]
Nếu $f$ liên tục trên $[a,b]$ và $f(a) \cdot f(b) < 0$ thì $\exists c \in (a,b): f(c) = 0$
\end{theorem}

\newpage

\section{CHƯƠNG III: PHÉP TÍNH VI PHÂN HÀM SỐ MỘT BIẾN SỐ}

\subsection{Đạo hàm của hàm số}

\subsubsection{Định nghĩa}

\begin{definition}[Đạo hàm tại một điểm]
Hàm $f$ khả vi tại $a$ nếu tồn tại giới hạn hữu hạn:
\[f'(a) = \lim_{h \to 0} \frac{f(a+h) - f(a)}{h} = \lim_{x \to a} \frac{f(x) - f(a)}{x - a}\]
Giới hạn này được gọi là đạo hàm của $f$ tại $a$, ký hiệu: $f'(a)$, $\frac{df}{dx}(a)$, $y'(a)$
\end{definition}

\textbf{Ý nghĩa hình học:} $f'(a)$ là hệ số góc của tiếp tuyến của đồ thị $y = f(x)$ tại điểm $(a, f(a))$.

Phương trình tiếp tuyến: $y - f(a) = f'(a)(x - a)$

\textbf{Ý nghĩa vật lý:} $f'(a)$ biểu thị vận tốc tức thời của chuyển động $s = f(t)$ tại thời điểm $t = a$.

\subsubsection{Bảng đạo hàm các hàm số cơ bản}

\begin{tcolorbox}[title=Bảng đạo hàm]
\begin{multicols}{2}
\begin{enumerate}
\item $(c)' = 0$
\item $(x^\alpha)' = \alpha x^{\alpha-1}$
\item $(a^x)' = a^x \ln a$
\item $(e^x)' = e^x$
\item $(\log_a x)' = \frac{1}{x\ln a}$
\item $(\ln x)' = \frac{1}{x}$
\item $(\sin x)' = \cos x$
\item $(\cos x)' = -\sin x$
\item $(\tan x)' = \frac{1}{\cos^2 x} = 1 + \tan^2 x$
\item $(\cot x)' = -\frac{1}{\sin^2 x} = -(1 + \cot^2 x)$
\item $(\arcsin x)' = \frac{1}{\sqrt{1-x^2}}$
\item $(\arccos x)' = -\frac{1}{\sqrt{1-x^2}}$
\item $(\arctan x)' = \frac{1}{1+x^2}$
\item $(\text{arccot}\, x)' = -\frac{1}{1+x^2}$
\end{enumerate}
\end{multicols}
\end{tcolorbox}

\subsubsection{Quy tắc tính đạo hàm}

\textbf{1. Đạo hàm của tổng, hiệu, tích, thương:}
\begin{align*}
(u \pm v)' &= u' \pm v'\\
(uv)' &= u'v + uv'\\
\left(\frac{u}{v}\right)' &= \frac{u'v - uv'}{v^2} \quad (v \neq 0)\\
(cu)' &= cu' \quad (c \text{ là hằng số})
\end{align*}

\textbf{2. Đạo hàm của hàm hợp:}
\[(f(g(x)))' = f'(g(x)) \cdot g'(x)\]

\textbf{3. Đạo hàm của hàm ngược:}
Nếu $y = f(x)$ có hàm ngược $x = f^{-1}(y)$ thì:
\[(f^{-1})'(y) = \frac{1}{f'(x)}\]

\subsection{Vi phân của hàm số}

\subsubsection{Định nghĩa}

\begin{definition}[Vi phân]
Vi phân của hàm $f$ tại $a$ là:
\[df(a) = f'(a) \cdot h\]
trong đó $h = dx$ là số gia của đối số.

Ta cũng viết: $df = f'(x)dx$ hay $dy = y'dx$
\end{definition}

\textbf{Tính chất:}
\begin{enumerate}
\item $d(u \pm v) = du \pm dv$
\item $d(uv) = vdu + udv$
\item $d\left(\frac{u}{v}\right) = \frac{vdu - udv}{v^2}$
\item $d(cu) = cdu$
\end{enumerate}

\subsection{Đạo hàm và vi phân cấp cao}

\subsubsection{Đạo hàm cấp cao}

\begin{definition}
Đạo hàm cấp $n$ của $f$, ký hiệu $f^{(n)}$, được định nghĩa quy nạp:
\begin{align*}
f^{(1)} &= f'\\
f^{(n)} &= (f^{(n-1)})', \quad n \geq 2
\end{align*}
\end{definition}

\textbf{Công thức Leibniz:}
\[(uv)^{(n)} = \sum_{k=0}^{n} C_n^k u^{(k)}v^{(n-k)}\]

\subsubsection{Vi phân cấp cao}

\begin{definition}
Vi phân cấp $n$:
\[d^n f = f^{(n)}(dx)^n\]
\end{definition}

\subsection{Các định lý về giá trị trung bình}

\subsubsection{Định lý Fermat}

\begin{theorem}[Fermat]
Nếu $f$ đạt cực trị tại $a$ và $f$ khả vi tại $a$ thì $f'(a) = 0$.
\end{theorem}

\subsubsection{Định lý Rolle}

\begin{theorem}[Rolle]
Nếu $f$ liên tục trên $[a,b]$, khả vi trên $(a,b)$ và $f(a) = f(b)$ thì tồn tại $c \in (a,b)$ sao cho $f'(c) = 0$.
\end{theorem}

\subsubsection{Định lý Lagrange}

\begin{theorem}[Lagrange - Định lý giá trị trung bình]
Nếu $f$ liên tục trên $[a,b]$ và khả vi trên $(a,b)$ thì tồn tại $c \in (a,b)$ sao cho:
\[f'(c) = \frac{f(b) - f(a)}{b - a}\]
\end{theorem}

\textbf{Hệ quả:}
\begin{itemize}
\item Nếu $f'(x) = 0, \forall x \in (a,b)$ thì $f$ là hàm hằng trên $(a,b)$
\item Nếu $f'(x) > 0, \forall x \in (a,b)$ thì $f$ tăng ngặt trên $(a,b)$
\item Nếu $f'(x) < 0, \forall x \in (a,b)$ thì $f$ giảm ngặt trên $(a,b)$
\end{itemize}

\subsubsection{Định lý Cauchy}

\begin{theorem}[Cauchy]
Nếu $f, g$ liên tục trên $[a,b]$, khả vi trên $(a,b)$ và $g'(x) \neq 0, \forall x \in (a,b)$ thì tồn tại $c \in (a,b)$ sao cho:
\[\frac{f'(c)}{g'(c)} = \frac{f(b) - f(a)}{g(b) - g(a)}\]
\end{theorem}

\subsection{Công thức Taylor và khai triển Maclaurin}

\subsubsection{Công thức Taylor}

\begin{theorem}[Taylor]
Nếu $f$ có đạo hàm đến cấp $n+1$ trên $(a,b)$ chứa điểm $x_0$ thì với mọi $x \in (a,b)$:
\[f(x) = \sum_{k=0}^{n} \frac{f^{(k)}(x_0)}{k!}(x-x_0)^k + R_n(x)\]
trong đó $R_n(x) = \frac{f^{(n+1)}(c)}{(n+1)!}(x-x_0)^{n+1}$ với $c$ nằm giữa $x_0$ và $x$.
\end{theorem}

\subsubsection{Công thức Maclaurin}

Trường hợp $x_0 = 0$:
\[f(x) = \sum_{k=0}^{n} \frac{f^{(k)}(0)}{k!}x^k + R_n(x)\]

\textbf{Khai triển Maclaurin các hàm thường gặp:}
\begin{align*}
e^x &= 1 + x + \frac{x^2}{2!} + \frac{x^3}{3!} + \ldots + \frac{x^n}{n!} + o(x^n)\\
\sin x &= x - \frac{x^3}{3!} + \frac{x^5}{5!} - \ldots + (-1)^n\frac{x^{2n+1}}{(2n+1)!} + o(x^{2n+1})\\
\cos x &= 1 - \frac{x^2}{2!} + \frac{x^4}{4!} - \ldots + (-1)^n\frac{x^{2n}}{(2n)!} + o(x^{2n})\\
\ln(1+x) &= x - \frac{x^2}{2} + \frac{x^3}{3} - \ldots + (-1)^{n-1}\frac{x^n}{n} + o(x^n)\\
(1+x)^\alpha &= 1 + \alpha x + \frac{\alpha(\alpha-1)}{2!}x^2 + \ldots + \frac{\alpha(\alpha-1)\ldots(\alpha-n+1)}{n!}x^n + o(x^n)
\end{align*}

\subsection{Quy tắc L'Hospital}

\begin{theorem}[L'Hospital]
Cho $\lim_{x \to a} f(x) = \lim_{x \to a} g(x) = 0$ (hoặc $= \infty$) và $g'(x) \neq 0$ trong lân cận của $a$.\\
Nếu $\lim_{x \to a} \frac{f'(x)}{g'(x)} = L$ thì $\lim_{x \to a} \frac{f(x)}{g(x)} = L$
\end{theorem}

\textbf{Các dạng vô định:} $\frac{0}{0}$, $\frac{\infty}{\infty}$, $0 \cdot \infty$, $\infty - \infty$, $0^0$, $1^\infty$, $\infty^0$

\subsection{Sự biến thiên và khảo sát hàm số}

\subsubsection{Tính đơn điệu}

\textbf{Điều kiện đủ để hàm đơn điệu:}
\begin{itemize}
\item Nếu $f'(x) > 0, \forall x \in (a,b)$ thì $f$ tăng ngặt trên $(a,b)$
\item Nếu $f'(x) < 0, \forall x \in (a,b)$ thì $f$ giảm ngặt trên $(a,b)$
\end{itemize}

\subsubsection{Cực trị}

\textbf{Điều kiện cần:} Nếu $f$ đạt cực trị tại $x_0$ và $f$ khả vi tại $x_0$ thì $f'(x_0) = 0$.

\textbf{Điều kiện đủ thứ nhất:}
\begin{itemize}
\item Nếu $f'(x)$ đổi dấu từ âm sang dương khi $x$ đi qua $x_0$ thì $f$ đạt cực tiểu tại $x_0$
\item Nếu $f'(x)$ đổi dấu từ dương sang âm khi $x$ đi qua $x_0$ thì $f$ đạt cực đại tại $x_0$
\end{itemize}

\textbf{Điều kiện đủ thứ hai:}
Giả sử $f'(x_0) = 0$ và $f''(x_0) \neq 0$:
\begin{itemize}
\item Nếu $f''(x_0) > 0$ thì $f$ đạt cực tiểu tại $x_0$
\item Nếu $f''(x_0) < 0$ thì $f$ đạt cực đại tại $x_0$
\end{itemize}

\subsubsection{Tính lồi, lõm và điểm uốn}

\begin{itemize}
\item Hàm $f$ lồi trên $(a,b)$ nếu $f''(x) > 0, \forall x \in (a,b)$
\item Hàm $f$ lõm trên $(a,b)$ nếu $f''(x) < 0, \forall x \in (a,b)$
\item Điểm $x_0$ là điểm uốn nếu $f''(x_0) = 0$ và $f''(x)$ đổi dấu khi $x$ đi qua $x_0$
\end{itemize}

\subsubsection{Tiệm cận}

\textbf{1. Tiệm cận đứng:} $x = a$ là tiệm cận đứng nếu:
\[\lim_{x \to a^+} f(x) = \pm\infty \quad \text{hoặc} \quad \lim_{x \to a^-} f(x) = \pm\infty\]

\textbf{2. Tiệm cận ngang:} $y = b$ là tiệm cận ngang nếu:
\[\lim_{x \to +\infty} f(x) = b \quad \text{hoặc} \quad \lim_{x \to -\infty} f(x) = b\]

\textbf{3. Tiệm cận xiên:} $y = ax + b$ là tiệm cận xiên nếu:
\[a = \lim_{x \to \pm\infty} \frac{f(x)}{x}, \quad b = \lim_{x \to \pm\infty} [f(x) - ax]\]

\newpage

\section{PHẦN BÀI TẬP ÔN TẬP}

\subsection{Bài tập Chương 1: Giới hạn của dãy số}

\subsubsection{Bài tập về số thực và số phức}

\textbf{Bài 1.} Chứng minh rằng $\sqrt{3}$ là số vô tỉ.

\textbf{Bài 2.} Tìm cận trên đúng và cận dưới đúng (nếu tồn tại) của tập:
\[E = \left\{\frac{1 + (-1)^n}{n^2} : n \in \mathbb{N}^*\right\}\]

\textbf{Bài 3.} Cho $z_1 = 2 + 3i$, $z_2 = 1 - i$. Tính:
\begin{enumerate}[a)]
\item $z_1 + z_2$
\item $z_1 \cdot z_2$
\item $\frac{z_1}{z_2}$
\item $|z_1|$, $|z_2|$
\end{enumerate}

\textbf{Bài 4.} Viết các số phức sau dưới dạng lượng giác và dạng mũ:
\begin{enumerate}[a)]
\item $z_1 = 1 + i$
\item $z_2 = -\sqrt{3} + i$
\item $z_3 = -1 - i\sqrt{3}$
\end{enumerate}

\textbf{Bài 5.} Sử dụng công thức Moivre tính:
\begin{enumerate}[a)]
\item $(1+i)^{10}$
\item $\left(\frac{1+i\sqrt{3}}{2}\right)^{20}$
\end{enumerate}

\subsubsection{Bài tập về giới hạn dãy số}

\textbf{Bài 6.} Bằng định nghĩa, chứng minh sự hội tụ và tìm giới hạn của các dãy:
\begin{enumerate}[a)]
\item $u_n = \frac{n}{n+1}$
\item $u_n = \frac{n-1}{4n+1}$
\item $u_n = \frac{3 + (-3)^n}{4^n}$
\end{enumerate}

\textbf{Bài 7.} Tính các giới hạn sau:
\begin{enumerate}[a)]
\item $\lim_{n \to \infty} \frac{3n^2 + 2n - 1}{5n^2 - n + 3}$
\item $\lim_{n \to \infty} \frac{2^n + 3^n}{2^n + 3^{n+1}}$
\item $\lim_{n \to \infty} (\sqrt{n+1} - \sqrt{n})$
\item $\lim_{n \to \infty} \sqrt[n]{n}$
\item $\lim_{n \to \infty} \left(1 + \frac{1}{n}\right)^n$
\end{enumerate}

\textbf{Bài 8.} Cho dãy số xác định bởi:
\[u_1 = 1, \quad u_{n+1} = \frac{1}{2}(u_n + 2)\]
Chứng minh dãy hội tụ và tìm giới hạn.

\textbf{Bài 9.} Xét sự hội tụ của các dãy số:
\begin{enumerate}[a)]
\item $u_n = \frac{1}{1 \cdot 2} + \frac{1}{2 \cdot 3} + \ldots + \frac{1}{n(n+1)}$
\item $u_n = 1 + \frac{1}{2^2} + \frac{1}{3^2} + \ldots + \frac{1}{n^2}$
\end{enumerate}

\textbf{Bài 10.} Chứng minh rằng:
\begin{enumerate}[a)]
\item $\lim_{n \to \infty} \frac{1^2 + 2^2 + \ldots + n^2}{n^3} = \frac{1}{3}$
\item $\lim_{n \to \infty} \frac{1 + 2 + \ldots + n}{n^2} = \frac{1}{2}$
\end{enumerate}

\subsection{Bài tập Chương 2: Hàm số một biến số}

\subsubsection{Bài tập về các khái niệm cơ bản}

\textbf{Bài 11.} Tìm tập xác định của các hàm số:
\begin{enumerate}[a)]
\item $y = \frac{1}{\sqrt{4-x^2}}$
\item $y = \ln(x^2 - 1)$
\item $y = \sqrt{x} + \frac{1}{\sqrt{1-x}}$
\item $y = \arcsin\left(\frac{x-1}{2}\right)$
\end{enumerate}

\textbf{Bài 12.} Xét tính chẵn, lẻ của các hàm số:
\begin{enumerate}[a)]
\item $y = x^2 + |x|$
\item $y = \frac{x}{x^2 - 1}$
\item $y = x^3 + \sin x$
\end{enumerate}

\textbf{Bài 13.} Tìm hàm ngược của các hàm số:
\begin{enumerate}[a)]
\item $y = 2x - 3$
\item $y = e^{x+1}$
\item $y = \frac{x+1}{x-1}$
\end{enumerate}

\subsubsection{Bài tập về giới hạn hàm số}

\textbf{Bài 14.} Tính các giới hạn:
\begin{enumerate}[a)]
\item $\lim_{x \to 2} \frac{x^2 - 4}{x - 2}$
\item $\lim_{x \to 1} \frac{x^3 - 1}{x^2 - 1}$
\item $\lim_{x \to \infty} \frac{3x^2 + 2x - 1}{2x^2 - x + 5}$
\item $\lim_{x \to 0} \frac{\sqrt{1+x} - 1}{x}$
\end{enumerate}

\textbf{Bài 15.} Tính các giới hạn lượng giác:
\begin{enumerate}[a)]
\item $\lim_{x \to 0} \frac{\sin 3x}{x}$
\item $\lim_{x \to 0} \frac{1 - \cos x}{x^2}$
\item $\lim_{x \to 0} \frac{\tan x - \sin x}{x^3}$
\item $\lim_{x \to 0} \frac{\sin x - x}{x^3}$
\end{enumerate}

\textbf{Bài 16.} Tính các giới hạn mũ và logarit:
\begin{enumerate}[a)]
\item $\lim_{x \to 0} \frac{e^x - 1}{x}$
\item $\lim_{x \to 0} \frac{\ln(1+x)}{x}$
\item $\lim_{x \to 0} \frac{a^x - 1}{x}$ $(a > 0)$
\item $\lim_{x \to \infty} \left(1 + \frac{1}{x}\right)^x$
\end{enumerate}

\textbf{Bài 17.} Tính các giới hạn dạng $0^0$, $1^\infty$, $\infty^0$:
\begin{enumerate}[a)]
\item $\lim_{x \to 0^+} x^x$
\item $\lim_{x \to 0} (1+x)^{\frac{1}{x}}$
\item $\lim_{x \to \infty} \left(\frac{x+1}{x-1}\right)^x$
\item $\lim_{x \to 0} (\cos x)^{\frac{1}{x^2}}$
\end{enumerate}

\subsubsection{Bài tập về tính liên tục}

\textbf{Bài 18.} Xét tính liên tục của các hàm số:
\begin{enumerate}[a)]
\item $f(x) = \begin{cases}
\frac{x^2-4}{x-2} & \text{nếu } x \neq 2\\
4 & \text{nếu } x = 2
\end{cases}$

\item $f(x) = \begin{cases}
\frac{\sin x}{x} & \text{nếu } x \neq 0\\
1 & \text{nếu } x = 0
\end{cases}$

\item $f(x) = \begin{cases}
x^2 & \text{nếu } x \leq 1\\
2x - 1 & \text{nếu } x > 1
\end{cases}$
\end{enumerate}

\textbf{Bài 19.} Tìm $a$ để hàm số liên tục tại $x = 0$:
\[f(x) = \begin{cases}
\frac{e^x - 1}{x} & \text{nếu } x \neq 0\\
a & \text{nếu } x = 0
\end{cases}\]

\textbf{Bài 20.} Chứng minh phương trình $x^3 - 3x + 1 = 0$ có ít nhất 3 nghiệm thực phân biệt.

\subsection{Bài tập Chương 3: Phép tính vi phân}

\subsubsection{Bài tập tính đạo hàm}

\textbf{Bài 21.} Tính đạo hàm của các hàm số:
\begin{enumerate}[a)]
\item $y = x^3 - 2x^2 + 5x - 1$
\item $y = (x^2 + 1)(x^3 - 2x)$
\item $y = \frac{x^2 - 1}{x + 1}$
\item $y = \sqrt{x^2 + 1}$
\item $y = e^{x^2}$
\item $y = \ln(x^2 + 1)$
\item $y = \sin(2x + 1)$
\item $y = x^x$ $(x > 0)$
\end{enumerate}

\textbf{Bài 22.} Tính đạo hàm của các hàm số hợp:
\begin{enumerate}[a)]
\item $y = e^{\sin x}$
\item $y = \ln(\cos x)$
\item $y = \sin^3(2x)$
\item $y = e^{x^2 + 1}$
\item $y = \arctan(e^x)$
\end{enumerate}

\textbf{Bài 23.} Viết phương trình tiếp tuyến của đồ thị hàm số:
\begin{enumerate}[a)]
\item $y = x^2 - 3x + 2$ tại điểm có hoành độ $x = 1$
\item $y = \sin x$ tại điểm có hoành độ $x = \frac{\pi}{4}$
\item $y = e^x$ tại điểm có hoành độ $x = 0$
\end{enumerate}

\textbf{Bài 24.} Tính đạo hàm cấp hai của các hàm số:
\begin{enumerate}[a)]
\item $y = x^4 - 2x^3 + x$
\item $y = e^{2x}$
\item $y = \sin(3x)$
\item $y = \ln x$
\end{enumerate}

\subsubsection{Bài tập về quy tắc L'Hospital}

\textbf{Bài 25.} Tính các giới hạn bằng quy tắc L'Hospital:
\begin{enumerate}[a)]
\item $\lim_{x \to 0} \frac{e^x - 1 - x}{x^2}$
\item $\lim_{x \to 0} \frac{\sin x - x}{\tan x - x}$
\item $\lim_{x \to 1} \frac{x^3 - 3x + 2}{x^3 - x^2 - x + 1}$
\item $\lim_{x \to \infty} \frac{\ln x}{x}$
\item $\lim_{x \to 0^+} x \ln x$
\end{enumerate}

\subsubsection{Bài tập về cực trị}

\textbf{Bài 26.} Tìm cực trị của các hàm số:
\begin{enumerate}[a)]
\item $y = x^3 - 3x + 2$
\item $y = x^4 - 4x^3 + 4x^2$
\item $y = \frac{x^2}{x-1}$
\item $y = x - \ln(1+x)$
\end{enumerate}

\textbf{Bài 27.} Tìm giá trị lớn nhất và nhỏ nhất của hàm số trên đoạn cho trước:
\begin{enumerate}[a)]
\item $y = x^3 - 3x + 1$ trên $[-2, 2]$
\item $y = \sin x + \cos x$ trên $[0, 2\pi]$
\item $y = x e^{-x}$ trên $[0, 3]$
\end{enumerate}

\subsubsection{Bài tập khảo sát hàm số}

\textbf{Bài 28.} Khảo sát và vẽ đồ thị các hàm số:
\begin{enumerate}[a)]
\item $y = x^3 - 3x^2 + 2$
\item $y = \frac{x^2 - 1}{x}$
\item $y = x^2 e^{-x}$
\end{enumerate}

\newpage

\section{ĐỀ THI THỬ}

\subsection{Đề thi thử số 1}

\textbf{Câu 1 (2 điểm):} Giới hạn\\
Tính các giới hạn sau:
\begin{enumerate}[a)]
\item $\lim_{n \to \infty} \frac{3n^2 + 2n - 5}{2n^2 - n + 1}$ (0.5 điểm)
\item $\lim_{x \to 2} \frac{x^3 - 8}{x^2 - 4}$ (0.5 điểm)
\item $\lim_{x \to 0} \frac{\sin 5x}{3x}$ (0.5 điểm)
\item $\lim_{x \to 0} \frac{e^{2x} - 1}{x}$ (0.5 điểm)
\end{enumerate}

\textbf{Câu 2 (2 điểm):} Đạo hàm bậc cao\\
Cho hàm số $y = x^3 e^x$. Tính:
\begin{enumerate}[a)]
\item $y'(x)$ (1 điểm)
\item $y''(x)$ (1 điểm)
\end{enumerate}

\textbf{Câu 3 (2 điểm):} Tích phân\\
Tính các tích phân sau:
\begin{enumerate}[a)]
\item $\int (3x^2 - 2x + 5)dx$ (1 điểm)
\item $\int x\sin x \, dx$ (1 điểm)
\end{enumerate}

\textbf{Câu 4 (2 điểm):} Khai triển Taylor\\
Khai triển hàm số $f(x) = e^x$ theo công thức Taylor đến cấp 3 tại điểm $x_0 = 0$.

\textbf{Câu 5 (2 điểm):} Fourier\\
Tìm chuỗi Fourier của hàm số $f(x) = x$ trên khoảng $(-\pi, \pi)$.

\subsection{Đáp án đề thi thử số 1}

\textbf{Câu 1:}
\begin{enumerate}[a)]
\item $\lim_{n \to \infty} \frac{3n^2 + 2n - 5}{2n^2 - n + 1} = \lim_{n \to \infty} \frac{3 + \frac{2}{n} - \frac{5}{n^2}}{2 - \frac{1}{n} + \frac{1}{n^2}} = \frac{3}{2}$

\item $\lim_{x \to 2} \frac{x^3 - 8}{x^2 - 4} = \lim_{x \to 2} \frac{(x-2)(x^2+2x+4)}{(x-2)(x+2)} = \lim_{x \to 2} \frac{x^2+2x+4}{x+2} = \frac{12}{4} = 3$

\item $\lim_{x \to 0} \frac{\sin 5x}{3x} = \frac{5}{3} \lim_{x \to 0} \frac{\sin 5x}{5x} = \frac{5}{3} \cdot 1 = \frac{5}{3}$

\item $\lim_{x \to 0} \frac{e^{2x} - 1}{x} = 2\lim_{x \to 0} \frac{e^{2x} - 1}{2x} = 2 \cdot 1 = 2$
\end{enumerate}

\textbf{Câu 2:}
\begin{enumerate}[a)]
\item $y' = (x^3)' e^x + x^3 (e^x)' = 3x^2 e^x + x^3 e^x = x^2e^x(3 + x)$

\item $y'' = (x^2e^x(3+x))' = (x^2)'e^x(3+x) + x^2(e^x)'(3+x) + x^2e^x(3+x)'$\\
$= 2xe^x(3+x) + x^2e^x(3+x) + x^2e^x = e^x(6x + 2x^2 + x^2(3+x) + x^2)$\\
$= e^x(6x + 2x^2 + 3x^2 + x^3 + x^2) = e^x(x^3 + 6x^2 + 6x)$
\end{enumerate}

\textbf{Câu 3:}
\begin{enumerate}[a)]
\item $\int (3x^2 - 2x + 5)dx = x^3 - x^2 + 5x + C$

\item $\int x\sin x \, dx$. Đặt $u = x, dv = \sin x dx$\\
$\Rightarrow du = dx, v = -\cos x$\\
$\int x\sin x \, dx = -x\cos x + \int \cos x \, dx = -x\cos x + \sin x + C$
\end{enumerate}

\textbf{Câu 4:}\\
$f(x) = e^x$, $f(0) = 1$\\
$f'(x) = e^x$, $f'(0) = 1$\\
$f''(x) = e^x$, $f''(0) = 1$\\
$f'''(x) = e^x$, $f'''(0) = 1$

Công thức Taylor:
\[e^x = 1 + x + \frac{x^2}{2!} + \frac{x^3}{3!} + o(x^3)\]

\textbf{Câu 5:}\\
Hàm $f(x) = x$ là hàm lẻ trên $(-\pi, \pi)$, nên $a_0 = 0$ và $a_n = 0$ với mọi $n$.

$b_n = \frac{1}{\pi}\int_{-\pi}^{\pi} x\sin(nx)dx = \frac{2}{\pi}\int_0^\pi x\sin(nx)dx$

Tích phân từng phần:
\[b_n = \frac{2}{\pi}\left[-\frac{x\cos(nx)}{n}\Big|_0^\pi + \frac{1}{n}\int_0^\pi \cos(nx)dx\right] = \frac{2}{\pi} \cdot \frac{-\pi\cos(n\pi)}{n} = \frac{-2\cos(n\pi)}{n} = \frac{2(-1)^{n+1}}{n}\]

Vậy chuỗi Fourier:
\[f(x) = \sum_{n=1}^{\infty} \frac{2(-1)^{n+1}}{n}\sin(nx) = 2\sin x - \sin 2x + \frac{2}{3}\sin 3x - \ldots\]

\newpage

\subsection{Đề thi thử số 2}

\textbf{Câu 1 (2 điểm):} Giới hạn\\
Tính các giới hạn sau:
\begin{enumerate}[a)]
\item $\lim_{x \to 1} \frac{x^2 - 1}{x - 1}$ (0.5 điểm)
\item $\lim_{x \to 0} \frac{1-\cos x}{x^2}$ (0.5 điểm)
\item $\lim_{x \to \infty} \left(1 + \frac{2}{x}\right)^x$ (0.5 điểm)
\item $\lim_{x \to 0} \frac{\ln(1+x)}{x}$ (0.5 điểm)
\end{enumerate}

\textbf{Câu 2 (2 điểm):} Đạo hàm bậc cao\\
Cho hàm số $y = \sin(2x)$. Tính:
\begin{enumerate}[a)]
\item $y'(x)$ (0.5 điểm)
\item $y''(x)$ (0.5 điểm)
\item $y'''(x)$ (0.5 điểm)
\item $y^{(4)}(x)$ (0.5 điểm)
\end{enumerate}

\textbf{Câu 3 (2 điểm):} Tích phân\\
Tính các tích phân sau:
\begin{enumerate}[a)]
\item $\int x^2 e^x dx$ (1 điểm)
\item $\int_0^{\pi} \sin^2 x \, dx$ (1 điểm)
\end{enumerate}

\textbf{Câu 4 (2 điểm):} Khai triển Taylor\\
Khai triển hàm số $f(x) = \ln(1+x)$ theo công thức Maclaurin đến cấp 4.

\textbf{Câu 5 (2 điểm):} Fourier\\
Tìm các hệ số $a_0, a_1, b_1$ trong khai triển Fourier của hàm số $f(x) = |x|$ trên khoảng $(-\pi, \pi)$.

\subsection{Đáp án đề thi thử số 2}

\textbf{Câu 1:}
\begin{enumerate}[a)]
\item $\lim_{x \to 1} \frac{x^2 - 1}{x - 1} = \lim_{x \to 1} \frac{(x-1)(x+1)}{x-1} = \lim_{x \to 1} (x+1) = 2$

\item $\lim_{x \to 0} \frac{1-\cos x}{x^2} = \lim_{x \to 0} \frac{2\sin^2(x/2)}{x^2} = \lim_{x \to 0} \frac{2\sin^2(x/2)}{4(x/2)^2} = \frac{2}{4} = \frac{1}{2}$

\item $\lim_{x \to \infty} \left(1 + \frac{2}{x}\right)^x = \lim_{x \to \infty} \left[\left(1 + \frac{2}{x}\right)^{x/2}\right]^2 = e^2$

\item $\lim_{x \to 0} \frac{\ln(1+x)}{x} = 1$ (giới hạn cơ bản)
\end{enumerate}

\textbf{Câu 2:}
\begin{enumerate}[a)]
\item $y' = 2\cos(2x)$
\item $y'' = -4\sin(2x)$
\item $y''' = -8\cos(2x)$
\item $y^{(4)} = 16\sin(2x)$
\end{enumerate}

\textbf{Câu 3:}
\begin{enumerate}[a)]
\item $\int x^2 e^x dx$. Tích phân từng phần hai lần:
\begin{align*}
\int x^2 e^x dx &= x^2e^x - 2\int xe^x dx\\
&= x^2e^x - 2(xe^x - \int e^x dx)\\
&= x^2e^x - 2xe^x + 2e^x + C\\
&= e^x(x^2 - 2x + 2) + C
\end{align*}

\item $\int_0^{\pi} \sin^2 x \, dx = \int_0^{\pi} \frac{1-\cos 2x}{2}dx = \frac{1}{2}\left[x - \frac{\sin 2x}{2}\right]_0^\pi = \frac{\pi}{2}$
\end{enumerate}

\textbf{Câu 4:}\\
$f(x) = \ln(1+x)$, $f(0) = 0$\\
$f'(x) = \frac{1}{1+x}$, $f'(0) = 1$\\
$f''(x) = -\frac{1}{(1+x)^2}$, $f''(0) = -1$\\
$f'''(x) = \frac{2}{(1+x)^3}$, $f'''(0) = 2$\\
$f^{(4)}(x) = -\frac{6}{(1+x)^4}$, $f^{(4)}(0) = -6$

Công thức Maclaurin:
\[\ln(1+x) = x - \frac{x^2}{2} + \frac{x^3}{3} - \frac{x^4}{4} + o(x^5)\]

\textbf{Câu 5:}\\
Hàm $f(x) = |x|$ là hàm chẵn, nên $b_n = 0$ với mọi $n$.

$a_0 = \frac{1}{2\pi}\int_{-\pi}^{\pi} |x|dx = \frac{1}{\pi}\int_0^\pi x\,dx = \frac{1}{\pi} \cdot \frac{\pi^2}{2} = \frac{\pi}{2}$

$a_1 = \frac{1}{\pi}\int_{-\pi}^{\pi} |x|\cos x \, dx = \frac{2}{\pi}\int_0^\pi x\cos x \, dx$

Tích phân từng phần:
\[a_1 = \frac{2}{\pi}[x\sin x\big|_0^\pi - \int_0^\pi \sin x \, dx] = \frac{2}{\pi}[0 - (-\cos x\big|_0^\pi)] = \frac{2}{\pi}[-((-1)-1)] = \frac{4}{\pi}\]

$b_1 = 0$ (do hàm chẵn)

\vspace{1cm}

\begin{center}
\end{center}

\end{document}
